\section{Discussion}\seclabel{conclusion}

It seems likely, given our results on the idealized tree-based
estimators from \secref{ideal} and \secref{convex}, that the greedy
tree-based estimators also behave well. In particular, we suspect that
our greedy tree-based estimators are minimax-optimal within
logarithmic factors. We leave this open to future work.

It is also often desirable for nonparametric estimators to be
adaptive, in the sense that they attain the optimal minimax rate
without depending on some of the important features of the
nonparametric class in question. In some cases, an adaptive density
estimate can be constructed by first estimating these features, and
then building a density estimate assuming the estimated features. For
example, in \cite{birge-risk}, an adapative estimate for
non-increasing densities is developed by first estimating the size of
the support, and plugging this estimated support size into a
non-adaptive estimate. We expect that in this manner, our method can
be made adaptive.

The techniques of this paper seem to naturally extend to higher
dimensions. Take, for instance, the class of block-decreasing
densities, whose minimax rate was identified by Biau and
Devroye~\cite{block}. This is the class of densities supported on
$[0, 1]^d$ bounded by some constant $B \ge 0$, such that each density
is non-increasing in each coordinate if all other coordinates are held
fixed. The discrete version of this class has each density supported
on $\{1, \dots, k\}^d$, with the monotonicity constraint. In order to
estimate such a density, one could devise an oriented binary splitting
rule analogous to \eqref{id-splitting-rule} and carry out a similar
analysis as performed in \secref{ideal}.
%In this case, we expect there
%again to be three regimes for the minimax risk:
%\begin{itemize}
%\item when $n \ll k^d$, the risk is constant;
%\item when $n \gg k^d$, the risk is proportional to $\sqrt{k^d/n}$, as
%  attained by the histogram estimate with bins of size $1$ for each
%  element of $\{1, \dots, k\}^d$, and;
%\item in an intermediate regime, the risk is some nonparametric rate.
%\end{itemize}

Furthermore, we expect that there are many other classes of
one-dimensional densities whose optimal minimax rate could be
identified using our approach, like the class of $\ell$-monotone
densities on $\{1, \dots, k\}$, where a function $f$ is called
\emph{$\ell$-monotone} if it is non-negative and if $(-1)^j f^{(j)}$
is non-increasing and convex for all $j \in \{0, \dots, \ell - 2\}$ if
$\ell \ge 2$, and where $f$ is non-negative and non-increasing if
$\ell = 1$. This paper tackles the cases of $\ell = 1$ and $\ell =
2$. Write $\sF_{k, \ell}$ for the class of $\ell$-monotone densities
on $\{1, \dots, k\}$. See Balabdaoui and Wellner~\cite{k-monotone,
  k-monotone-2} for texts concerning the density estimation of
$\ell$-monotone densities. It seems likely that our method could be
applied to prove the following conjecture.
\begin{conj}
Let $f \colon \N \times \N \times \N \times \R \to \R$ be
\[
 f(n, k, \ell, \alpha) = \left\{
                         \begin{array}{ll}
                                \sqrt{k/n} & \mbox{if $2 \le k \le \alpha n^{\frac{1}{2\ell + 1}}$,} \\
                                {\left(\frac{\log_\alpha (k/n^{\frac{1}{2\ell + 1}})}{n}\right)}^{\frac{\ell}{2\ell + 1}} & \mbox{if $\alpha n^{\frac{1}{2\ell + 1}} \le k \le n^{\frac{1}{2\ell + 1}} \alpha^n$,} \\
                                1 & \mbox{if $n^{\frac{1}{2\ell + 1}} \alpha^n \le k.$}
                         \end{array}\right.
\]
Let $\ell \ge 1$ be fixed. There are constants $\alpha, C, n_0 \ge 1$
depending only on $\ell$ such that, for $n \ge n_0$,
\[
  \frac{1}{C} \le \frac{\sR_n(\sF_{k, \ell})}{f(n, k, \ell, \alpha)} \le C .
\]
\end{conj}
The main obstacle in proving the above would be the development of
good local estimates for $\ell$-monotone densities, in the same flavor
as \propref{id-tv} and \propref{conv-id-tv}.

Our approach also likely can be applied to the class of all
log-concave discrete distributions, where we recall that $f : \N \to
[0, 1]$ is called \emph{log-concave} if
\[
  f(x) f(x + 2) \le f(x + 1)^2 , \quad \text{for all } x \ge 1 .
\]
See \cite{kane-logconcave-up, samworth-lower, samworth-survey} for a
small selection of works on the density estimation of $d$-dimensional
log-concave continuous densities. The optimal Hellinger distance
minimax rate (within logarithmic factors) for this class was recently
obtained by Dagan and Kur~\cite{dagan}, who showed that it is attained
by the maximum-likelihood estimate. There remains a small gap between
the best known upper and lower bounds in the TV-distance minimax rate
as of the time of writing.

\section*{Acknowledgments}

We would like to thank the three reviewers and an associate editor for
their helpful comments and suggestions.
